\setbeamercolor{background canvas}{bg=fitblue}
\begin{frame}
\frametitle{Základní návod}
\begin{center}
\Huge {\color{white}Základní návod}
\end{center}
\end{frame}
\setbeamercolor{background canvas}{bg=white}

\begin{frame}[fragile]
\frametitle{Základní návod}
	\begin{itemize}
	\item Složka obsahující CMakeLists.txt je složka, kterou je možné sestavit
  \item Hierarchie složek
  \item CMake sám o sobě nesestavuje aplikace/knihovny
  \item CMake slouží pro generování makefile / sln souborů
  \item CMake obsahuje aplikace \textbf{cmake cmake-gui ccmake ctest cpack}
  \item \textbf{cmake} - rovnou vygeneruje makefile (parametery můžou ovlivnit generování
  \item \textbf{cmake-gui} / \textbf{ccmake} - uživatelské nastavní parametrů a následné vygenerování makefile
  \item \textbf{ctest} - pro spouštění testů (unit tests, ...)
  \item \textbf{cpack} - pro vytvoření balíčků (*.deb, ...)
	\end{itemize}
\end{frame}

\setbeamercolor{background canvas}{bg=fitblue}
\begin{frame}
\frametitle{Prázdný CMake}
\begin{center}
\Huge {\color{white}Prázdný CMake}
\end{center}
\end{frame}
\setbeamercolor{background canvas}{bg=white}

\begin{frame}[fragile]
\frametitle{Prázdný CMakeLists.txt}
CMakeLists.txt:
\inputminted[frame=lines]{cmake}{../examples/00-EmptyCMake/CMakeLists.txt}
\end{frame}

\setbeamercolor{background canvas}{bg=fitblue}
\begin{frame}
\frametitle{HelloWorld CMake}
\begin{center}
\Huge {\color{white}HelloWorld CMake}
\end{center}
\end{frame}
\setbeamercolor{background canvas}{bg=white}

\begin{frame}[fragile]
\frametitle{HelloWorld CMakeLists.txt}
CMakeLists.txt:
\inputminted[frame=lines]{cmake}{../examples/01-HellowWorld/CMakeLists.txt}
main.cpp:
\inputminted[frame=lines]{c++}{../examples/01-HellowWorld/main.cpp}
\end{frame}

\begin{frame}[fragile]
\frametitle{HelloWorld Sestavení}
CMake umožňuje out-of-source building - vytváření *.o souborů mimo *.cpp soubory.
{\small
\begin{Verbatim}[commandchars=\\\{\}]
$ \textbf{ls}
CMakeLists.txt main.cpp
$ \textbf{mkdir} build
$ \textbf{cd} build
$ \textbf{cmake} ..
$ \textbf{ls}
CMakeCache.txt  \textcolor{blue}{CMakeFiles}  cmake_install.cmake  Makefile
$ \textbf{make}
$ \textbf{ls}
CMakeCache.txt  \textcolor{blue}{CMakeFiles}  cmake_install.cmake  \textcolor{orange}{HelloWorld}  Makefile
\end{Verbatim}
}
\end{frame}

\setbeamercolor{background canvas}{bg=fitblue}
\begin{frame}
\frametitle{Include Directories}
\begin{center}
\Huge {\color{white}Include Directories}
\end{center}
\end{frame}
\setbeamercolor{background canvas}{bg=white}

\begin{frame}[fragile]
\frametitle{IncludeDirectories CMakeLists.txt}
  CMakeLists.txt:
  {\scriptsize \inputminted[frame=lines]{cmake}{../examples/02-IncludeDirectories/CMakeLists.txt}}
  src/main.cpp:
  {\scriptsize \inputminted[frame=lines]{c++}{../examples/02-IncludeDirectories/src/main.cpp}}
  include/header.h:
  {\scriptsize \inputminted[frame=lines]{c++}{../examples/02-IncludeDirectories/include/header.h}}
\end{frame}

\setbeamercolor{background canvas}{bg=fitblue}
\begin{frame}
\frametitle{Compile Definitions}
\begin{center}
\Huge {\color{white}Compile Definitions}
\end{center}
\end{frame}
\setbeamercolor{background canvas}{bg=white}

\begin{frame}[fragile]
\frametitle{Compile Definitions}
  CMakeLists.txt:
  {\scriptsize \inputminted[frame=lines]{cmake}{../examples/03-CompileDefinitions/CMakeLists.txt}}
  src/main.cpp:
  {\scriptsize \inputminted[frame=lines]{c++}{../examples/03-CompileDefinitions/main.cpp}}
\end{frame}

\setbeamercolor{background canvas}{bg=fitblue}
\begin{frame}
\frametitle{Sestavení a instalace knihovny třetí strany}
\begin{center}
\Huge {\color{white}Sestavení a instalace knihovny třetí strany}
\end{center}
\end{frame}
\setbeamercolor{background canvas}{bg=white}

\begin{frame}[fragile]
\frametitle{Sestavení a instalace knihovny třetí strany}
\url{https://github.com/dormon/Vars}
{\scriptsize
\begin{Verbatim}[commandchars=\\\{\}]
$ \textbf{tree}
\textcolor{blue}{.}
|
|-- CMakeLists.txt
|-- main.cpp
|-- \textcolor{blue}{Vars}     \textcolor{gray}{# extern library}
    |-- CMakeLists.txt
    |-- README.md
    |-- \textcolor{blue}{src}
    |   |-- \textcolor{blue}{Vars}
    |       |-- Fwd.h
    |       |-- Resource.cpp
    |       |-- Resource.h
    |       |-- Vars.cpp
    |       |-- Vars.h
    |       |-- VarsImpl.cpp
    |       |-- VarsImpl.h
    |-- \textcolor{blue}{tests}
        |-- catch.hpp
        |-- CMakeLists.txt
        |-- tests.cpp
\end{Verbatim}
}
\end{frame}

\begin{frame}[fragile]
\frametitle{Sestavení a instalace knihovny třetí strany}
{\small
\begin{Verbatim}[commandchars=\\\{\}]
$ \textbf{mkdir} varsBuild
$ \textbf{mkdir} install
$ \textcolor{gray}{# vygenerování makefile}
$ \textcolor{gray}{# -H složka s CMakeLists.txt}
$ \textcolor{gray}{# -B složka pro build}
$ \textcolor{gray}{# -D nastavení parameteru (proměnné)}
$ \textcolor{gray}{# \textcolor{red}{CMAKE_INSTALL_PREFIX} místo pro instalaci knihovny}
$ \textbf{cmake} -H\textcolor{blue}{Vars} -B\textcolor{blue}{varsBuild/} -D\textcolor{red}{CMAKE_INSTALL_PREFIX}=\textcolor{blue}{install/}
$ \textcolor{gray}{# spuštění překladu a instalace}
$ \textcolor{gray}{# --build cesta k build složce}
$ \textcolor{gray}{# --target název cíle, který se bude sestavovat}
$ \textcolor{gray}{# pokud cíl závisí na jiných, nejprve se sestaví ty}
$ \textbf{cmake} --build \textcolor{blue}{varsBuild/} --target install
\end{Verbatim}
}  
\end{frame}

\begin{frame}[fragile]
\frametitle{Sestavení a instalace knihovny třetí strany}
Po instalaci vypadá složka s nainstalovanou knihovnou takto:
\begin{Verbatim}[commandchars=\\\{\}]
$ \textbf{tree} \textcolor{blue}{install/}
|-- \textcolor{blue}{include}
|   |-- \textcolor{blue}{Vars}
|       |-- Fwd.h
|       |-- Resource.h
|       |-- vars_export.h
|       |-- Vars.h
|-- \textcolor{blue}{lib}
    |-- \textcolor{blue}{cmake}
    |   |-- \textcolor{blue}{Vars}
    |       |-- VarsConfig.cmake
    |       |-- VarsConfigVersion.cmake
    |       |-- VarsTargets.cmake
    |       |-- VarsTargets-noconfig.cmake
    |-- libVars.a
\end{Verbatim}
\end{frame}

\setbeamercolor{background canvas}{bg=fitblue}
\begin{frame}
\frametitle{Využití nainstalované knihovny}
\begin{center}
\Huge {\color{white}Využití nainstalované knihovny}
\end{center}
\end{frame}
\setbeamercolor{background canvas}{bg=white}

\begin{frame}[fragile]
\frametitle{Využití nainstalované knihovny}
main.cpp:
{\small \inputminted[frame=lines]{c++}{../examples/04-BuildingAndInstalling/main.cpp}}
CMakeLists.txt:
{\small \inputminted[frame=lines]{cmake}{../examples/04-BuildingAndInstalling/CMakeLists.txt}}
\end{frame}

\begin{frame}[fragile]
\frametitle{Využití nainstalované knihovny}
\begin{Verbatim}[commandchars=\\\{\}]
$ \textbf{ls}
CMakeLists.txt  \textcolor{blue}{install}  main.cpp  \textcolor{blue}{Vars}  \textcolor{blue}{varsBuild}
$ \textbf{mkdir} build
$ \textcolor{gray}{# vygenerování makefile pro aplikaci}
$ \textcolor{gray}{# využívající knihovnu Vars}
$ \textcolor{gray}{# \textcolor{red}{Vars_DIR} je proměnná, která by měla ukazovat na}
$ \textcolor{gray}{# složku, obsahující soubor VarsConfig.cmake}
$ \textbf{cmake} -H\textcolor{blue}{.} -B\textcolor{blue}{build/} -D\textcolor{red}{Vars_DIR}=\textcolor{blue}{install/lib/cmake/Vars}
$ \textcolor{gray}{# sestavení aplikace}
$ \textbf{cmake} --build \textcolor{blue}{build/}
\end{Verbatim}
\end{frame}

